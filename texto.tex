% vim: set tw=80 ts=2 sts=2 sw=2 et:
\documentclass[12pt]{article}

\usepackage{mathtools}
\usepackage{amsfonts}
\usepackage{hyperref}

\title{Trabalho 2 de Geometria Computacional}
\author{Gustavo Silveira Frehse GRR20203927}
\begin{document}
  \maketitle
  \section{O Problema}
  Dados os $n$ vértices de um polígono, queremos gerar uma triangulação dele.
  
  \section{Algoritmo}
  Usamos o algoritmo de cortar orelhas, descrito em \cite{earcut} e com tempo
  $O(n^2)$. 

  Utilizamos uma DCEL para representar o polígono, assim temos a operação de
  adicionar diagonais em tempo constante.
  
  \subsection{Pré-processamento}
  O algoritmo consiste primeiramente em pré-processar os vértices, guardando se
  cada vértice é uma orelha.
  
  Para descobrir se um $v_i$ vértice é uma orelha, para todos vértices $v_j$ no
  polígono checamos se $v_j$ está no triângulo formado por $v_{i-1}, v_i,
  v_{i+1}$, sendo $v_{i-1}$ e $v_{i+1}$ os vizinhos de $v_i$. Se $v_j$ está
  dentro do triângulo, $v_i$ não é orelha. Assim essa operação é $O(n)$.
  
  Como fazemos isso para todos os vértices, o pré-processamento é $O(n^2)$.
  
  \subsection{Cortar Orelhas}
  No algoritmo, caminhamos pelos vértices, se $v_i$ é orelha, adicionamos a
  diagonal $v_{i-1}, v_{i+1}$. Como os únicos vértices que podem mudar seus
  status como orelha são $v_{i-1}$ e $v_{i+1}$, recalculamos se $v_{i-1}$ e
  $v_{i+1}$ são orelha.
  
  Como o número de diagonais é $n-3$, e para cada diagonal calculamos duas vezes
  se é um vértice é orelha, o algoritmo é $O((n - 3) * (2n)) = O(n^2)$.

  \begin{thebibliography}{9}
    \bibitem{earcut} O'Rourke, J. (1998). Computational Geometry in C (2nd ed.).
  \end{thebibliography}


\end{document}
